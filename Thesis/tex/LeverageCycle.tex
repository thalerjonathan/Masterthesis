\documentclass[../Bachelorarbeit.tex]{subfiles}
\begin{document}

\chapter{The Leverage Cycle}
\label{ch:leverageCycle}

TODO: optimism factor which denotes the probability an agents assigns the event of up-state tomorrow

Definition des Modells
		Märkte, Marktmechanismen, clearing, utiliy funktionen,....
		alles theoretisch, um des dann in implementierung praktisch zu zeigen
	Bestehende Resultate mit Bezug auf paper
		Fully-Connected: prozess und endverteilung, erreicht theoretisches Gleichgewicht approximativ

\section{Collateral Equilibrium}
\label{sec:collateralEquilibrium}

\begin{center}
$i_{1} = \frac{q - 0.2}{V - 0.2}$
\end{center}

\begin{center}
$i_{2} = \frac{0.2(p - q)}{0.8q - (V - 0.2)p}$
\end{center}

\begin{center}
$p = \frac{1}{i_{1}} - 1$
\end{center}

\begin{center}
$q = p \frac{i_{2} - i{1}}{1 - i_{1}}$
\end{center}

\end{document}
