\documentclass[Bachelorarbeit.tex]{subfiles}
\begin{document}
\chapter{Conclusion, Summary and further Research}

\section{Conclusion}

\section{Summary}

\section{Further Research}

\subsection{In-depth analysis of market-activities}
je nach markt-aktivität kommt sicher ein anderes gleichgewicht heraus bzw. ist das so?
hier bedarf es sicher weiterer foschungen und ist sicher auch ein ergiebiges und interessantes thema.

\subsection{Imporance-Sampling}
importance-sampling allgemein

\subsection{Experiments with real subjects}
experimentelle simulationen mit echten menschen: einschränken der handelsbeziehungen
wie lokal bzw. global muss die vernetzung sein (ascending-connected full shortcuts)

\subsection{Mathematical proof of hypothesis}
beweisbarkeit der ascending-connected (MIT/OHNE neuem Markt)

\subsection{Equilibrium definition for continous double-auction process}

\end{document}