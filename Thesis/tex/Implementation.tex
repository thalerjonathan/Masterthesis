\documentclass[Bachelorarbeit.tex]{subfiles}
\begin{document}
\chapter{Implementation}
\label{ch:implementation}

\section{Requirements}
Wieso Java? Es existierte ja bereits ein "simulationsprogramm" in C++ vom Breuer-Team auf das ich zugriff hatte.

\section{Functionality}

\subsection{Inspection}

\subsection{Replications}

\subsection{Experiments}
\subsubsection{GUI}
\subsubsection{Command-Line}

\section{Architecture}
\subsection{Frontend}
\subsection{Controller}
\subsection{Backend}

\section{Agents}
zentrale klasse, die extrem viel kapselt.

\section{Markets}
\subsection{Asset/Cash}
TODO same as in "A new market"
\subsubsection{Price-Range}
\subsubsection{Bid-Offering}
\subsubsection{Ask-Offering}
\subsubsection{Match}

\subsection{Loan/Cash}
TODO same as in "A new market"
\subsubsection{Price-Range}
\subsubsection{Bid-Offering}
\subsubsection{Ask-Offering}
\subsubsection{Match}

\subsection{Asset/Loan}
TODO same as in "A new market"
\subsubsection{Price-Range}
\subsubsection{Bid-Offering}
\subsubsection{Ask-Offering}
\subsubsection{Match}

\section{Simulation}
\subsection{Sweeping and Matching}

\section{Performance improvement}
Matching Wahrscheinlichkeiten
Importance Sampling
Lokales vs Globales Offerbook
		
\section{Calculating theoretical Equilibrium}
TODO analyse matlab-script of martin jandacka

\end{document}