\documentclass[Bachelorarbeit.tex]{subfiles}
\begin{document}
\chapter*{Abstract}
In the paper of \cite{Breuer2015} a model for endogenous leverage in a continuous double-auction is introduced and it is shown under which circumstances holdings and trading prices approach an equilibrium. One main criteria is the trading network the agents use where Breuer et al. examine only two topologies and report that the prices come to an equilibrium only in the case of a fully connected network. They leave the question open on how the model behaves with different kind of networks and which network topology exactly allows an equilibrium to be reached  for further research. This thesis builds upon this model and gives a hypothesis for a necessary condition a network must satisfy to allow the model to approach an equilibrium. Then a few network-topologies are examined in regard of their ability to allow equilibria to be reached or not through computer-driven simulation. As will be shown in this thesis the hypothesis turns out to be correct both through validation by computer-driven simulation and through mathematical proof.

In "Introduction" I will give a deeper insight on the workings of the model of Breuer et al. and introduce the hypothesis.

In "State of the art" an overview of abstract networks and their properties is given. Also network-generating algorithms are presented and discussed. Because continuous double-auctions are the type of market which is used for matchings a short introduction is given on this topic too.

In "Analysis" the hypothesis is verified and examined thoroughly and a mathematical proof is given.

In "Development" I will give an overview on the implementation of the computer-driver simulation of Breuer et al and the extensions introduced by this thesis which are network-generating algorithms, real-time visualisations and tracking of a whole process over time. Also a short overview of the software-architecture is given.

In "Implementation" the experiments, their setup and design are discussed.

In "Evaluation" the results of the previously discussed experiments are shown.

In "Interpretation" the results of the experiments are discussed.

\end{document}