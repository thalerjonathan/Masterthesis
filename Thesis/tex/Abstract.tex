\documentclass[Bachelorarbeit.tex]{subfiles}
\begin{document}
\chapter*{Abstract}
In the paper of \cite{Breuer2015} a model for endogenous leverage in a continuous double-auction is introduced and it is shown under which circumstances holdings and trading prices approach an equilibrium. One main criteria is the trading network the agents use where Breuer et al. examine only two topologies and report that the prices come to an equilibrium only in the case of a fully connected network. They leave the question open on how the model behaves with different kind of networks and which network topology exactly allows an equilibrium to be reached  for further research. This thesis builds upon this model and gives a hypothesis for a necessary condition a network must satisfy to allow the model to approach an equilibrium. Then a few network-topologies are examined in regard of their ability to allow equilibria to be reached or not through computer-driven simulation. As will be shown in this thesis through validation by computer-driven simulation the hypothesis turns out to be correct only after extending the simulation-model by an additional market. This result raises questions this thesis tries to solve about market-mechanisms and market-types when agents don't trade in a fully informed network. 

In "Introduction" a short motivation for the development of such models as in \cite{Breuer2015} is given and what the goal and purpose of such research is.

In "Theory" the theoretical background involved with this thesis is presented. First Leverage and systemic risk and its implications are discussed. Then an introduction into the mechanics of Continuous Double Auction as market-mechanisms and equilibrium theory in economics is given. Finally an overview of abstract networks, network-generationg algorithms and and their properties is given.

In "The Leverage Cycle" the theoretical model \cite{Breuer2015} built their simulation upon is discussed in-depth.

In "Hypothesis" the conjecture about the type of topology necessary to reach the theoretical equilibirum is presented and discussed in comparison with a range of various other topologies including small-world and scale-free ones.

"Implementation" gives an in-depth explanation of the implementation of the computer-driven simulation presented in \cite{Breuer2015} including a description of the architecture, implementation of the markets and trading mechanisms.

"Results" concentrates the content of the previous chapters to show that the initial conjecture/hypothesis does not satisfy the equlibrium and shows how it can be reached by introducing an additional market. Then results using this market are given and discussed where will be shown that using the additional market an equilibrium will be reached but that it is different from the theoretical predictions.

In "Conclusions" a short sum-up of the thesis and questions left for further research are presented.

\end{document}