\documentclass[Bachelorarbeit.tex]{subfiles}
\begin{document}

\graphicspath{{./figures/hypothesis/}}	%specifying the folder for the figures

\chapter{Hypothesis}
\label{ch:hypothesis}

In this chapter the question of the importance of fully-connectedness for reaching the equilibrium is raised where the question is whether it is really necessary to have a fully-connected network to reach equilibrium or not. We challenge this and claim that a much lower connectivity with a special property is sufficient. First the motivation is presented and the claims behind it are proven mathematically. Then the property is introduced and it is proven that it is necessary to reach theoretical equilibrium. Whether it is sufficient is tested by computer-driven simulation where the results are given in chapter \ref{ch:results}.

\medskip

The initial hypothesis presented in this chapter was conjectured first by the supervisor of this thesis Mr. Hans-Joachim Vollbrecht.

\section{Motivation}
The motivation behind the hypothesis is the fact that according to the double-auction definition - see chapter \ref{ch:theory} - for a match to happen the buyer-price must be larger or equal to the seller-price. This can only be the case if the buyer has a higher optimism-factor than the seller because only then the limit-price of the buyer will be larger than the one of the seller. For a match to occur the limit-price of the buyer has to be strictly larger than the one of the seller as shown below.

\medskip

In figure \ref{fig:MATCHING_BUYER_SELLER_RANGES} the price-ranges of both a seller and buyer are given where limit(s) and limit(b) denote the limit-price of the seller and buyer respectively, which are determined by their optimism-factor \textit{h}. The seller places its offerings in the price-range of [limit(s)..pU] as it wants to sell the goods above the expected price to make a profit. The buyer places its offerings in the price-range of [pD..limit(b)] as it wants to buy the goods below the expected price to make a profit. The resulting matching-range on which the prices can meet - again buyer-price $\geq$ seller-price - is marked by the red rectangle. It is easy to see that a match with these mechanics can occur only if the optimism-factor of the buyer limit(b) is strictly higher than the one of the seller limit(s).

\begin{figure}[H]
	\centering
  \includegraphics[width=1.0\textwidth, angle=0]{MATCHING_BUYER_SELLER_RANGES.png}
  	\caption{Matching of buyers and sellers price-ranges. The red rectangle marks the matching-range.}
	\label{fig:MATCHING_BUYER_SELLER_RANGES}
\end{figure}

\section{Proof buyer more optimistic than seller}
Proving that the optimism of a buyer must be greater than or equal to the optimism of the seller $h_B \geq h_S$ by using the fact that the limit-price of the buyer must be greater than or equal to the limit-price of the seller $limit_B \geq limit_S$ and then showing that this can only be the case if $h_B \geq h_S$.

\subsection{Asset/Cash}
$limit_B = h_B \, pU + (1-h_B) \, pD = h_B + \frac{1}{5} - \frac{h_B}{5}$ \\
$limit_S = h_S \, pU + (1-h_S) \, pD = h_S + \frac{1}{5} - \frac{h_S}{5}$

\begin{proof}
\begin{align*}
	limit_S - limit_B \leq 0
	\\ (h_S + \frac{1}{5} - \frac{h_S}{5}) - ( h_B + \frac{1}{5} - \frac{h_B}{5} ) \leq 0
	\\ h_S + \frac{1}{5} - \frac{h_S}{5} - h_B - \frac{1}{5} + \frac{h_B}{5} \leq 0
	\\ 5h_S + 1 - h_S - 5h_B - 1 + h_B \leq 0
	\\ 4h_S - 4h_B \leq 0
	\\ h_S - h_B \leq 0		\tag*{can only hold if $h_B \geq h_S$}
\end{align*}
\end{proof}

\subsection{Bond/Cash}
$limit_B = h_B \, V + (1-h_B) \, pD = h_B \, V + \frac{1}{5} - \frac{h_B}{5}$ \\
$limit_S = h_S \, V + (1-h_S) \, pD = h_S \, V + \frac{1}{5} - \frac{h_S}{5}$

\begin{proof}
\begin{align*}
	limit_S - limit_B \leq 0
	\\ (h_S \, V  + \frac{1}{5} - \frac{h_S}{5}) - ( h_B \, V  + \frac{1}{5} - \frac{h_B}{5} ) \leq 0
	\\ h_S \, V  + \frac{1}{5} - \frac{h_S}{5} -  h_B \, V  - \frac{1}{5} + \frac{h_B}{5} \leq 0
	\\ 5 h_S \, V  + 1 - h_S - 5h_B \, V  - 1 + h_B \leq 0
	\\ 5 h_S \, V - 5h_B \, V - h_S + h_B \leq 0
	\\ 5V(h_S - h_B) - (h_S - h_B) \leq 0	
	\\ (h_S - h_B)(5V - 1) \leq 0			\tag*{$|: (5V-1) \Rightarrow \geq 0 \; | \; V \; [0..1]$}
	\\ h_S - h_B \leq 0
\end{align*}
\end{proof}

\pagebreak

\subsection{Asset/Bond}
$limit_B = \frac{h_B + \frac{1}{5} - \frac{h_B}{5}}{h_B \, V + \frac{1}{5} - \frac{h_B}{5}}$ \\
$limit_S = \frac{h_S + \frac{1}{5} - \frac{h_S}{5}}{h_S \, V + \frac{1}{5} - \frac{h_S}{5}}$ \\

\begin{proof}
\begin{align*}
	limit_S - limit_B \leq 0
	\\ \frac{h_S + \frac{1}{5} - \frac{h_S}{5}}{h_S \, V + \frac{1}{5} - \frac{h_S}{5}} - \frac{h_B + \frac{1}{5} - \frac{h_B}{5}}{h_B \, V + \frac{1}{5} - \frac{h_B}{5}} \leq 0
	\\ (h_S + \frac{1}{5} - \frac{h_S}{5})(h_B \, V + \frac{1}{5} - \frac{h_B}{5}) - (h_B + \frac{1}{5} - \frac{h_B}{5})(h_S \, V + \frac{1}{5} - \frac{h_S}{5}) \leq 0													
	\\ substituting \; S = \frac{1}{5} - \frac{h_S}{5}, B = \frac{1}{5} - \frac{h_B}{5}
	\\ (h_S + S)(h_B \, V + B) - (h_B + B)(h_S \, V + S) \leq 0
	\\ h_S \, h_B \, V + h_S \, B + S \, h_B \, V + BS - (h_B \, h_S \, V + h_B \, S + B \, h_S \, V + BS) \leq 0
	\\ h_S \, h_B \, V + h_S \, B + S \, h_B \, V + BS - h_B \, h_S \, V - h_B \, S - B \, h_S \, V - BS \leq 0
	\\ h_S \, B + S \, h_B \, V - h_B \, S - B \, h_S \, V \leq 0
	\\ h_S \, B - h_B \, S + V(h_B \, S - h_S \, B ) \leq 0 
	\\ h_S(\frac{1}{5} - \frac{h_B}{5}) - h_B(\frac{1}{5} - \frac{h_S}{5}) + V(h_B(\frac{1}{5} - \frac{h_S}{5}) - h_S(\frac{1}{5} - \frac{h_B}{5})) \leq 0
	\\ \frac{h_S}{5} - \frac{h_S \, h_B}{5} - \frac{h_B}{5} + \frac{h_B \, h_S}{5} + \frac{h_B \, V}{5} - \frac{h_B \, h_S \, V}{5} - \frac{h_S \, V}{5} + \frac{h_S \, h_B \, V}{5} \leq 0 
	\\ \frac{h_S}{5} - \frac{h_B}{5} + \frac{h_B \, V}{5} - \frac{h_S \, V}{5} \leq 0
	\\ h_S - h_B + h_B \, V - h_S \, V \leq 0 
	\\ h_S( 1 - V ) + h_B( -1 + V ) \leq 0 
	\\ |: (1-V) \Rightarrow \geq 0 \; | \; V \; [0..1]
	\\ h_S + h_B \frac{-(1-V)}{(1-V)} \leq 0
	\\ h_S - h_B \leq 0
\end{align*}
\end{proof}

\section{Proof of monotony of limit-functions}
The property that the optimism-factor of the buyer has to be greater than or equal to the one of the seller is not enough for a match to occur. Additionally the limit-function must be monotonically increasing in the defined range [0..1] of the optimism-factor \textit{h} because if it is not then no matching would be possible. It is proven by showing that $limit' > 0$ in the range of \textit{h} = [0..1].

\subsection{Asset/Cash market}
\begin{proof}
\begin{align*}
	limit_{asset} = h \, pU + (1-h) pD		\tag*{pU = 1, pD = 0.2}
	\\ = h + (1 - h)0.2
	\\ = h + \frac{1}{5} - \frac{h}{5}		\tag*{$\frac{\mathrm d}{\mathrm d h}$}
	\\ = 1 - \frac{1}{5}
		&= \frac{4}{5}
\end{align*}
Constant $\Rightarrow$ limit-function is monotonically increasing over the real numbers. QED
\end{proof}

\subsection{Bond/Cash market}
\begin{proof}
\begin{align*}
	limit_{bond} = h \, V + (1-h) pD			
	\\ = h \, V + (1 - h)pD
	\\ = h \, V + pD - h\,pD		\tag*{$\frac{\mathrm d}{\mathrm d h}$}
	\\ = V - pD
\end{align*}
V is a constant in range of [0..1] $\Rightarrow$ limit-function is monotonically increasing were $V > pD$. QED
\end{proof}

Note that this implies that a bond with face-value \textit{V = pD} cannot be traded as the limit-function is constant for all \textit{h}. This needs to be covered in the implementation.

\subsection{Asset/Bond market}
\begin{proof}
\begin{align*}
	limit_{asset/bond} = \frac{h \, pU + (1-h) pD}{h \, V + (1-h) pD} 				\tag*{pU = 1, pD = 0.2}
	\\ = \frac{h + \frac{1}{5} - \frac{h}{5}}{h \, V + \frac{1}{5} - \frac{h}{5}}	\tag*{$\frac{\mathrm d}{\mathrm d h}$}
	\\ = - \frac{5(V-1)}{(h(5V-1)+1)^2}									\tag*{assume h and V in range [0..1]}
	\\ \Rightarrow 5(V-1) \leq 0
	\\ \Rightarrow (h(5V-1)+1)^2 \geq 0
	\\ \Rightarrow - \frac{5(V-1)}{(h(5V-1)+1)^2} \geq 0 			\tag*{for h and V in range [0..1]}
\end{align*}
first derivation $> 0$ $\Rightarrow$ limit-function is monotony increasing if $h > 0$ and $V > 0$. QED
\end{proof}

Note that the limit-function is undefined in the case of h=0 which should be prevented in the implementation - see chapter \ref{ch:implementation}.

\section{Hypothesis}
\begin{equation}
\begin{split}
\textit{fully-connectedness equilibrium} \iff \\
\forall \: \textit{agent-pairs} \: (a_{1},a_{n}) \: \exists \:  \textit{path} \: P \: \{a_{1}, a_{2}, ... , a_{n-1}, a_{n}\} \: | \: h(a_{i}) < h(a_{i+1})
\end{split}
\end{equation}

\newtheorem{conj}{Conjecture}
\begin{conj}
If and only if for all agents exists a path between two agents in which each visited agent has a larger optimism factor than the previous one then the same equilibrium as in fully-connectedness will be reached.
\end{conj}

\section{Proof of necessity}
In this section it is proven that the given property is \textit{necessary} to reach theoretical equilibrium but it is open to question whether it is \textit{sufficient} or not. This will be investigated by computer-simulation.

\subsection{Most minimal topology satisfying hypothesis}
As proven above the buyer optimism must be larger than the seller optimism. Each agent acts both as a buyer and a seller thus the optimism-factor \textit{h} is creating a total ordering on the agents where $h_i \leq h_{i+1}$. When now the agent with $h_i$ is connected to $h_{i+1}$ a Hamiltonian path is created which is per definition a spanning tree because it visits each vertex exactly once. If we now assign to each edge the weight of 1, it is clear to see that this graph must also be the minimal spanning tree. An example for such a graph is the Ascending-Connected topology - see appendix \ref{app:topologies} - which is the major network of interest in this thesis (besides the fully-connected one) as it is the most minimal topology which satisfies the property of the hypothesis.

\begin{figure}[H]
	\centering
  \includegraphics[width=1.0\textwidth, angle=0]{MINIMAL_NETWORK_OPTIMISM.png}
  	\caption{The minimal graph which satisfies the property of the hypothesis is a Hamiltonian path between the totally ordered agents.}
	\label{fig:MINIMAL_NETWORK_OPTIMISM}
\end{figure}

\subsection{Violation of property}
To show that the property of the hypothesis is necessary to reach equilibrium it is now shown that a violation of the property must lead to a miss-allocation because of inability to trade, thus missing equilibrium.

\medskip

A failure in reaching equilibrium is equivalent in miss-allocation of goods to agents who don't value the goods as much as others would. Trading with another agent could lead to an increased utility of both agents but trading is impossible due to a violation of the property through an invalid network. To see when and how miss-allocations could occur one must investigate the flow-direction of goods.

\paragraph{Asset/Cash} In this market the buyer $h_{i+1}$ gets assets and gives cash to the seller $h_i$. Thus the flow is as follows:
\begin{center}
$Asset: h_i \to h_{i+1}$
\end{center}
\begin{center}
$Cash: h_i \gets h_{i+1}$
\end{center}

\paragraph{Bond/Cash} In this market the buyer $h_{i+1}$ grants a loan to the seller by giving cash to the seller $h_i$ which takes a bond from the buyer. Thus the flow is as follows:
\begin{center}
$Bond: h_i \to h_{i+1}$
\end{center}
\begin{center}
$Cash: h_i \gets h_{i+1}$
\end{center}

\paragraph{Asset/Bond} In this market the buyer $h_{i+1}$ gets assets and takes a bond from seller $h_i$. Thus the flow is as follows:
\begin{center}
$Asset: h_i \to h_{i+1}$
\end{center}
\begin{center}
$Bond: h_i \gets h_{i+1}$
\end{center}

\medskip

There are basically 3 violations possible within the most minimal network. Note that also combinations of those violations are possible but for simplicity each is treated separately.

\begin{figure}[H]
	\centering
  \includegraphics[width=1.0\textwidth, angle=0]{OPTIMISM_VIOLATION_SELLER.png}
  	\caption{There is no connection between the seller $h_{i+1}$ and the buyer $h_{i+2}$ thus there exists no Hamlitonian path in the ordered agents which violates the property.}
	\label{fig:OPTIMISM_VIOLATION_SELLER}
\end{figure}

In figure \ref{fig:OPTIMISM_VIOLATION_SELLER} the seller $h_{i+1}$ is unable to sell to $h_i$ because of the lower optimism-factor and is unable to sell to $h_{i+2}$ obviously because of a non-existent connection. Depending on whether this violation happens in the pessimists, medianists or optimists different miss-allocations will happen.

\begin{itemize}
\item Asset/Cash - $h_{i+1}$ buys assets against cash but cannot sell them further up to $h_{i+2}$ because of the missing link. This is a problem in the range of pessimists as they try to get minimal on assets and maximal on cash. For medianists this is a problem too as they also try to get minimal on assets but maximal on bonds. Optimists try to get as many assets as possible thus no miss-allocation will show up there.
\item Bond/Cash market - $h_{i+1}$ buys bonds but cannot sell them further up to $h_{i+2}$ because of the missing link. This is a problem in the range of pessimists as they try to get minimal on bonds but maximal on assets. This is also a problem in the range of optimimsts.
\item Asset/Bond market - $h_{i+1}$ buys assets against bonds but cannot sell them further up to $h_{i+2}$ because of the missing link. This is a problem in the range of pessimists and medianists as they try to get maximally short on assets.
\end{itemize}

\begin{figure}[H]
	\centering
  \includegraphics[width=1.0\textwidth, angle=0]{OPTIMISM_VIOLATION_BUYER.png}
  	\caption{There is no connection between the buyer $h_{i+2}$ and the seller $h_{i+1}$ thus there exists no Hamlitonian path in the ordered agents which violates the property.}
	\label{fig:OPTIMISM_VIOLATION_BUYER}
\end{figure}

In figure \ref{fig:OPTIMISM_VIOLATION_BUYER} the buyer $h_{i+2}$ is unable to buy from $h_{i+1}$ because of a non-existent connection. Depending on whether this violation happens in the pessimists, medianists or optimists different miss-allocations happen.

\begin{itemize}
\item Asset/Cash - $h_{i+2}$ sells assets up to $h_{i+3}$ transforming to cash but can't use the cash to buy goods because of the missing link to seller $h_{i+1}$. This is a problem in the range of optimists as they try to maximise their assets and minimize their cash. Its also a problem in the range of medianists as they try to maximise their bonds and minimize their cash. Pessimists try to sell all assets and get as much cash as possible thus no miss-allocation will show up there.
\item Bond/Cash market - $h_{i+2}$ sells bonds up to $h_{i+3}$ transforming to cash but can't use the cash because of the missing link to seller $h_{i+1}$. This is the same situation as in Asset/Cash market.
\item Asset/Bond market - $h_{i+2}$ sells assets against bonds up to $h_{i+3}$ transforming assets to bonds but can't use the bonds to buy goods because of the missing link to seller $h_{i+1}$. This is the same situation as in Asset/Cash market.
\end{itemize}

\begin{figure}[H]
	\centering
  \includegraphics[width=1.0\textwidth, angle=0]{OPTIMISM_VIOLATION_DISCONNECTED.png}
  	\caption{Graph is disconnected.}
	\label{fig:OPTIMISM_VIOLATION_DISCONNECTED}
\end{figure}

In figure \ref{fig:OPTIMISM_VIOLATION_DISCONNECTED} obviously the buyer $h_{i+2}$ is unable to trade with the seller $h_{i+1}$ because of a non-existent connection thus it is trivial to see that theoretical equilibrium cannot be reached.

\section{Predictions}
The following topologies found in appendix \ref{app:topologies} satisfy the definition of the hypothesis:

\begin{itemize}
\item Fully-Connected
\item Half-Fully connected
\item Ascending-Connected
\item Ascending-Connected with all kind of short-cuts
\item Erdos-Renyi and Watts-Strogatz with the correct parametrization by pure chance.
\end{itemize}

If the hypothesis is valid and the property is sufficient these topologies will reach the equilibrium found in fully-connectedness. All other topologies do not satisfy the necessary property and are expected to clearly fail reaching the equilibrium of the Fully-Connected topology.

\medskip

See chapter \ref{ch:results} and \ref{ch:interpretation} where it is explained whether the property is sufficient or not. 

\end{document}