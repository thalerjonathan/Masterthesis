\documentclass[../Bachelorarbeit.tex]{subfiles}
\begin{document}

\chapter{State of the art and Theory on Networks}
TODO:
In "State of the art" an overview of abstract networks and their properties is given. Also network-generating algorithms are presented and discussed. Because continuous double-auctions are the type of market which is used for matchings a short introduction is given on this topic too.\\

TODO: der theorie-teil. Soll in die verwendete Theorie des Hauptteils einführen und darauf hinweisen, aber nicht völlig trocken und losgelöst vom hauptteil sein. Soll immer den kontext des hauptteils berücksichtigen und schon gewisse anwendungsfälle vorwegnehmen.

\section{Complex Networks}
TODO: ziel hier eine theoretische übersicht über  netzwerk-theorie zu geben wobei hauptaugenmerk auf die entwicklungen der letzten jahre (scale-free, small-world, ...)

Regular Graphs: \citep[vgl.]{BarabasiAlbert_StatisticalMechanics} \citep[vgl.]{Newman_ComplexNetworks}

Random Graphs: but since then, most large scale networks with no apparent design principle were described as random graphs introduced by two Hungarian mathematicians Paul Erdos and Alfred Renyi \citep[vgl.]{ErdosRenyi_RandomGraphs} \citep[vgl.]{ErdosRenyi_EvolutionRandomGraphs} Have small-world properties.

Small World Graphs or Average Path Length: Stanley Milgram \citep{TraverMilgram_StudySmallWorld} \citep{Milgram_SmallWorld} \citep{Kleinberg_SmallworldAlgorithmic}

Clustering Coefficient or Transitivity \citep{WattsStrogatz_DynamicsSmallWorld}

Degree Distribution \citep{BarabasiAlbert_EmergenceScaling} Generally, it was believed that the degree distribution in most networks follows a
Poisson distribution but in reality, real world networks have a highly skewed degree distribution following power-laws. Power-laws are expressions of the form y / x, where is a constant, x and y are the measures of interest [152].

Small World and Scale Free Network: A small world network as deined by Watts and Strogatz \cite{WattsStrogatz_DynamicsSmallWorld}, is a network with high clustering coeffcient and small average path length. A scale free network as deined by Barabasi and Albert \citep{BarabasiAlbert_EmergenceScaling}, is a network where the degree distribution follows a power law.

Complex Networks: are Small-World and/or Scale-Free \citep{BarratWeigt_PropertiesSmallWorld} \citep{AmaralScalaStanley_ClassesSmallWorld}

\citep {Kleinfeld_BigWorld}
http://www.cs.princeton.edu/~chazelle/courses/BIB/big-world.htm

Mathematical stuff 
\citep{Newman_Eigenvectors}
\citep{AielloChungLu_RandomEvolution}
\citep{EbelMielschBornholdt_TopologyEmail}
\citep{GaertlerPatrignani_AutonomousSystem}

\section{Network-Generating Algorithms}
- fully connected
- ascending connected
- ascending connected with shortcuts
- hubs
- erdos-renyi
- barbasi-albert
- watts-strogatz

\section{Continuous Double-Auctions}
auszüge aus dem Breuer et al. Paper und Everything you wanted to know about Continous Double-Auctions
\end{document}