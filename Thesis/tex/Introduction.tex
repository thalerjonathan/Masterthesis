\documentclass[Bachelorarbeit.tex]{subfiles}

\begin{document}
\chapter{Introduction}
In 2008 the so called \textit{Subprime Mortgage Crisis} struck the world. It was caused by declining house prices which rose to an all-time high during the US Housing Market Bubble in 2006. Borrowers used their asset as collateral for the mortgage which constantly increased in value which guaranteed them a low payment-rate because the rate was coupled to the value of the asset. Banks granted "subprime" mortgages to more and more highly risky borrowers. In 2007 borrowers started to default which led to falling prices as the banks reclaimed the collateral and wanted to sell it again on the market to compensate for the loss. This led to a flood of assets on the market which led to a decline of housing prices overall. As the prices fell dramatically the payment-rates rose dramatically to compensate for the cheaper asset. This in turn resulted in even more borrowers going default because of margin requirements which resulted in a dramatic downward spiral. Even worse the banks were selling these collateralized products to each other and even insured themselves against defaults of borrowers which led to an even more dramatic kick-back. \cite{SubprimeExplainedA}, \cite{SubprimeExplainedB}
\medskip
The primary driving force behind systemic risk in the aftermath of the \textit{Subprime Mortgage Crisis} was identified as leverage which is the mechanism of borrowing money to buy goods which in turn act as a security for the borrowed money.

\section{Motivation}
In the classic economics literature leverage was always an exogenous parameter in the works on collateralized credit but recently \cite{Geanakoplos2009} and \cite{GeanakoplosZame2014} proposed an equilibrium framework which endogenized leverage. \cite{Breuer2015} built upon this findings and developed a simulation on top of the equilibrium frameworks in which zero-intelligence agents trade assets and bonds in a continuous double-auction. Breuer et al. asked 

\begin{quote}
... whether the competitive theory of trade in leveraged assets has descriptive and predictive power in a double auction environment.
\end{quote}

and wanted to better understand the dynamic of such an equilibrium process, how prices develop and whether they approach the equilibrium predicted in the framework or not. They made three contributions:

\begin{enumerate}
\item Continuous double-auction for leveraged assets is new.
\item Institutional specification matter a lot.
\item Robustness tests were conducted to show under which circumstances equilibrium cannot be reached.
\end{enumerate}

The authors could show that in their simulation trading prices and wealth-distribution approach the theoretical equilibrium of \cite{Geanakoplos2009}. They investigated a fully connected network and a hub-network of agents where the equilibrium was only reached in the case of the fully connected network. 

\section{Objectives}
This thesis investigates additional topologies of networks and their convergence towards theoretical equilibrium. Furthermore it presents a hypothesis which properties a topology of a network must satisfy to reach the same equilibrium as in \cite{Breuer2015}. For experimental investigation a software was built for this thesis which implemented the exact simulation model of \cite{Breuer2015} but extended it further to be applicable to arbitrary topologies.

\section{Struture}
The thesis starts with chapter \ref{ch:theory} where the theoretical background involved with this thesis is handled. In chapter \ref{ch:leverageCycle} a short overview of the model which is used in the simulation of this thesis is presented. In chapter \ref{ch:hypothesis} the hypothesis which is the motivation for this thesis is derived. Chapter \ref{ch:implementation} then gives an in-depth explanation of the implementation of the computer-driven simulation built for this thesis. Chapter \ref{ch:results} validates the previously described thesis-software against the results found in \cite{Breuer2015} and reports the results for the additional network-topologies. Chapter \ref{ch:interpretation} connects the content of the previous chapters to verify whether the hypothesis of chapter \ref{ch:hypothesis} is valid or not. The hypothesis is not valid and needs an adjustment in the form of a new market which is introduced and discussed with additional results in chapter \ref{ch:newMarket}. Chapter \ref{ch:conclusions} gives conclusions on the findings of the thesis and outlines topics for further research.

\end{document}