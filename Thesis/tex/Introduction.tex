\documentclass[Bachelorarbeit.tex]{subfiles}
\begin{document}
\chapter{Introduction}
TODO:
In "Introduction" I will give a deeper insight on the workings of the model of Breuer et al. and introduce the hypothesis.

das orginalpaper kurz vorstellen, dass dort wenig nachforschungen bezüglich netzwerken gemacht wurde. Diese thesis geht den netzwerken auf den grund, aber nur als rein abstrakte netzwerke.
das modell vorstellen, was gehandelt wird, zero-intelligence agents, utility-funktionen erklären, equilibrium erklären.

hypothese vorstellen: jedes paar von agenten muss über einen kantenzug erreichbar sein, in dem der optimismusfaktor von agent zu agent monoton wächst.


aim of the paper:
leverage in markets and how it affects prices of assets.
motivation: understanding financial crisis because leverage was identified "as one of the key drivers of systemic risk"
builds upon the paper of geanakoplos and zame and geanakoplos to "endogenzie leverage within a general equilibrium framework"
"endogenous leverage also has asset pricing implications: In general, both equilibrium asset prices and the prices of debt are distorted."
trading mechanism used: continous double auctions (see State-Of-The-Art for details on continous double auctions)
agent-based simulation with blind utility improvers.
paper unique in that it presents double auctions for leveraged assets.
	-> how is collateral cleared and how is leveraged purchases coordinated across asset and debt markets
analyse final prices and robustness, when trade is restricted to neighbours then final prices different!

the model:
up and down state of the world tomorrow
one unit of cash, one unit of assets
in up-state asset 1.0, down asset is 0.2 worth
klarstellen, dass der up/down state nie wirklich ausgewürfelt wird, dieses experiment ist einstufig.



\end{document}