\documentclass[Bachelorarbeit.tex]{subfiles}
\begin{document}
\chapter{Conclusions}
\label{ch:conclusions}

The motivation for this thesis was to investigate the influence of various types of network-topologies in continuous double-auctions on the convergence towards theoretical equilibrium. It builds upon findings and results of \cite{Breuer2015} and an equilibrium framework developed by \cite{Geanakoplos2009}. As a starting point a hypothesis was formulated which gives properties a network must exhibit to reach the theoretical equilibrium defined in the model. Interpreting the results of the thesis-software it turned out that the hypothesis is not feasible for the model of \cite{Breuer2015} as the Ascending-Connected topology - the most minimal topology which satisfies the properties postulated by the hypothesis - leads to serious miss-allocation of wealth in the range of the pessimist agents. The hope then was to be able to repair the miss-allocation by extending the original model by introducing a new market and that then theoretical equilibrium will be reached. This new market on which collateral can be traded against cash is able to remove the allocation inefficiencies but the equilibrium is still different from the theoretical one thus the hypothesis turns out to be invalid and infeasible for the extended model too.

\bigskip

The major conclusion on the findings of the thesis is that equilibrium seems only to be possible in a fully-connected trading-network. Thus as soon as a market-institution exhibits signs of restricting trades between agents to a reduced neighbourhood then it will lead to miss-allocation and fail reaching equilibrium and thus will be unfair. The introduction of a new market solves the miss-allocations in theory but to the best knowledge of the author of this thesis such a market does not exist yet and it is open to question whether such a market can be put into practice in the real world and whether agents accept its products and the impact it has on the real trading world.

\section*{Further Research}

\subsection*{Importance-Sampling}
Importance sampling was used to increase the matching-probabilities for the special case of an Ascending-Connected network which led to a dramatic performance enhancement as much less matching-rounds were required to reach the point where no trading was possible any more. Further research could investigate a more general mechanisms of increasing the matching-probabilities independent of the topology without introducing a bias. One way could be through sampling the price-ranges in different intervals or learning and adapting them on-the-fly during a simulation run.

\subsection*{In-depth analysis of market-activities}
The market-activities as presented in chapter \ref{ch:interpretation} and to a greater extend in chapter \ref{ch:newMarket} are only covered by intuitive description what is going on and why - no theory about market-activity is presented. Further research could dedicate its attention to develop a serious theory why the activities behave as they do, how they interact between each other and how they influence equilibrium. Furthermore it could establish connections to market-activity theory in economics to validate the simulation-results against theoretical frameworks.

\subsection*{Experiments with real subjects}
Experimental economics has a long history and allows to verify theoretical equilibrium frameworks or theories. The author of this thesis is part of a group which has conducted a pilot experiment with real subjects on the subject of trading collateralized assets based on the model of \cite{Breuer2015}. This was to best knowledge the first time that an experiment in this field was ever undertaken. Although it was only a pilot it gave already valuable insights on the models of \cite{Breuer2015} and \cite{Geanakoplos2009} but more experiments need to be carried out to be able to give serious results. Based upon the results and findings of this thesis the already existing trading software could be extended furthermore to restrict the trading between agents to a specific topology e.g. Ascending-Connected. It would be of much interest whether intelligent human traders are able to resolve the miss-allocations produced by the Ascending-Connected topology without the use of the new market and if they can approach theoretical equilibrium.

\subsection*{Equilibrium theory vs. equilibrium process}
It was noted on multiple occasions in this thesis that there is a fundamental difference between the equilibrium theory upon the model of the thesis is built and the equilibrium process implemented in the thesis-software. Further research could look for a formal definition of equilibrium in such a process and combines it with the given equilibrium theory of complex systems in general and economics in particular to find a definition for the equilibrium of such a trading process.

\end{document}