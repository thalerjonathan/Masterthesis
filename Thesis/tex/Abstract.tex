\documentclass[Bachelorarbeit.tex]{subfiles}
\begin{document}
\chapter*{Abstract}
In the paper of \cite{Breuer2015} a model for endogenous leverage in a continuous double-auction is introduced and it is shown under which circumstances holdings and trading prices approach an equilibrium. One main criteria is the trading network the agents use where \cite{Breuer2015} examine only two topologies and report that the prices come to an equilibrium only in the case of a fully connected network. They leave the question open on how the model behaves with different kind of networks and which network topology exactly allows an equilibrium to be reached  for further research. This thesis builds upon this model and gives a hypothesis for the necessary property a network must satisfy to allow the model to approach theoretical equilibrium as reported in \cite{Breuer2015}. Then a few network-topologies are examined in regard of their ability to allow equilibria to be reached or not through computer-driven simulation. As will be shown in this thesis through validation by computer-driven simulation the hypothesis turns out to be correct only after extending the model by an additional market. Thus this thesis answers questions about market-mechanisms and market-types when agents don't trade in a fully informed network. 
\end{document}